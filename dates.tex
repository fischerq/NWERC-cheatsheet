\subsection{Dates}
\begin{lstlisting}[language=C++]
// Routines for performing computations on dates.  In these routines,
// months are expressed as integers from 1 to 12, days are expressed
// as integers from 1 to 31, and years are expressed as 4-digit
// integers.

string dayOfWeek[] = {"Mon", "Tue", "Wed", "Thu", "Fri", "Sat", "Sun"};

// converts Gregorian date to integer (Julian day number)
int dateToInt (int m, int d, int y){  
  return 
    1461 * (y + 4800 + (m - 14) / 12) / 4 +
    367 * (m - 2 - (m - 14) / 12 * 12) / 12 - 
    3 * ((y + 4900 + (m - 14) / 12) / 100) / 4 + 
    d - 32075;
}

// converts integer (Julian day number) to Gregorian date: month/day/year
void intToDate (int jd, int &m, int &d, int &y){
  int x, n, i, j;
  
  x = jd + 68569;
  n = 4 * x / 146097;
  x -= (146097 * n + 3) / 4;
  i = (4000 * (x + 1)) / 1461001;
  x -= 1461 * i / 4 - 31;
  j = 80 * x / 2447;
  d = x - 2447 * j / 80;
  x = j / 11;
  m = j + 2 - 12 * x;
  y = 100 * (n - 49) + i + x;
}

// converts integer (Julian day number) to day of week
string intToDay (int jd){
  return dayOfWeek[jd % 7];
}

int main (int argc, char **argv){
  int jd = dateToInt (3, 24, 2004);
  int m, d, y;
  intToDate (jd, m, d, y);
  string day = intToDay (jd);
  
  // expected output:
  //    2453089
  //    3/24/2004
  //    Wed
  cout << jd << endl
    << m << "/" << d << "/" << y << endl
    << day << endl;
}
\end{lstlisting}
