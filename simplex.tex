\subsection{Simplex Algorithm}
\begin{lstlisting}

// Two-phase simplex algorithm for solving linear programs of the form
//
//     maximize     c^T x
//     subject to   Ax <= b
//                  x >= 0
//
// INPUT: A -- an m x n matrix
//        b -- an m-dimensional vector
//        c -- an n-dimensional vector
//        x -- a vector where the optimal solution will be stored
//
// OUTPUT: value of the optimal solution (infinity if unbounded
//         above, nan if infeasible)
//
// To use this code, create an LPSolver object with A, b, and c as
// arguments.  Then, call Solve(x).

#include <iostream>
#include <iomanip>
#include <vector>
#include <cmath>
#include <limits>

using namespace std;

typedef long double DOUBLE;
typedef vector<DOUBLE> VD;
typedef vector<VD> VVD;
typedef vector<int> VI;

const DOUBLE EPS = 1e-9;

struct LPSolver {
  int m, n;
  VI B, N;
  VVD D;

  LPSolver(const VVD &A, const VD &b, const VD &c) : 
    m(b.size()), n(c.size()), N(n+1), B(m), D(m+2, VD(n+2)) {
    for (int i = 0; i < m; i++) for (int j = 0; j < n; j++) D[i][j] = A[i][j];
    for (int i = 0; i < m; i++) { B[i] = n+i; D[i][n] = -1; D[i][n+1] = b[i]; }
    for (int j = 0; j < n; j++) { N[j] = j; D[m][j] = -c[j]; }
    N[n] = -1; D[m+1][n] = 1;
  }
	   
  void Pivot(int r, int s) {
    for (int i = 0; i < m+2; i++) if (i != r)
      for (int j = 0; j < n+2; j++) if (j != s)
	D[i][j] -= D[r][j] * D[i][s] / D[r][s];
    for (int j = 0; j < n+2; j++) if (j != s) D[r][j] /= D[r][s];
    for (int i = 0; i < m+2; i++) if (i != r) D[i][s] /= -D[r][s];
    D[r][s] = 1.0 / D[r][s];
    swap(B[r], N[s]);
  }

  bool Simplex(int phase) {
    int x = phase == 1 ? m+1 : m;
    while (true) {
      int s = -1;
      for (int j = 0; j <= n; j++) {
	if (phase == 2 && N[j] == -1) continue;
	if (s == -1 || D[x][j] < D[x][s] || D[x][j] == D[x][s] && N[j] < N[s]) s = j;
      }
      if (D[x][s] >= -EPS) return true;
      int r = -1;
      for (int i = 0; i < m; i++) {
	if (D[i][s] <= 0) continue;
	if (r == -1 || D[i][n+1] / D[i][s] < D[r][n+1] / D[r][s] ||
	    D[i][n+1] / D[i][s] == D[r][n+1] / D[r][s] && B[i] < B[r]) r = i;
      }
      if (r == -1) return false;
      Pivot(r, s);
    }
  }

  DOUBLE Solve(VD &x) {
    int r = 0;
    for (int i = 1; i < m; i++) if (D[i][n+1] < D[r][n+1]) r = i;
    if (D[r][n+1] <= -EPS) {
      Pivot(r, n);
      if (!Simplex(1) || D[m+1][n+1] < -EPS) return -numeric_limits<DOUBLE>::infinity();
      for (int i = 0; i < m; i++) if (B[i] == -1) {
	int s = -1;
	for (int j = 0; j <= n; j++) 
	  if (s == -1 || D[i][j] < D[i][s] || D[i][j] == D[i][s] && N[j] < N[s]) s = j;
	Pivot(i, s);
      }
    }
    if (!Simplex(2)) return numeric_limits<DOUBLE>::infinity();
    x = VD(n);
    for (int i = 0; i < m; i++) if (B[i] < n) x[B[i]] = D[i][n+1];
    return D[m][n+1];
  }
};

int main() {

  const int m = 4;
  const int n = 3;  
  DOUBLE _A[m][n] = {
    { 6, -1, 0 },
    { -1, -5, 0 },
    { 1, 5, 1 },
    { -1, -5, -1 }
  };
  DOUBLE _b[m] = { 10, -4, 5, -5 };
  DOUBLE _c[n] = { 1, -1, 0 };
  
  VVD A(m);
  VD b(_b, _b + m);
  VD c(_c, _c + n);
  for (int i = 0; i < m; i++) A[i] = VD(_A[i], _A[i] + n);

  LPSolver solver(A, b, c);
  VD x;
  DOUBLE value = solver.Solve(x);
  
  cerr << "VALUE: "<< value << endl;
  cerr << "SOLUTION:";
  for (size_t i = 0; i < x.size(); i++) cerr << " " << x[i];
  cerr << endl;
  return 0;
}


\end{lstlisting}