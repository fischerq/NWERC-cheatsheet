\subsection{Knuth-Morris-Pratt Algorithm}
\begin{lstlisting}[language=C++]
/* Searches for the string w in the string s (of length k). Returns the 0-based
index of the first match (k if no match is found).
Algorithm runs in O(k) time. */

typedef vector<int> VI;

void buildTable(string& w, VI& t)
{
  t = VI(w.length());  
  int i = 2, j = 0;
  t[0] = -1; t[1] = 0;
  
  while(i < w.length())
  {
    if(w[i-1] == w[j]) { t[i] = j+1; i++; j++; }
    else if(j > 0) j = t[j];
    else { t[i] = 0; i++; }
  }
}

int KMP(string& s, string& w)
{
  int m = 0, i = 0;
  VI t;
  
  buildTable(w, t);  
  while(m+i < s.length())
  {
    if(w[i] == s[m+i])
    {
      i++;
      if(i == w.length()) return m;
    }
    else
    {
      m += i-t[i];
      if(i > 0) i = t[i];
    }
  }  
  return s.length();
}

int main()
{
  string a = (string) "The example above illustrates the general technique for assembling "+
    "the table with a minimum of fuss. The principle is that of the overall search: "+
    "most of the work was already done in getting to the current position, so very "+
    "little needs to be done in leaving it. The only minor complication is that the "+
    "logic which is correct late in the string erroneously gives non-proper "+
    "substrings at the beginning. This necessitates some initialization code.";
  
  string b = "table";
  
  int p = KMP(a, b);
  cout << p << ": " << a.substr(p, b.length()) << " " << b << endl;
}

\end{lstlisting}