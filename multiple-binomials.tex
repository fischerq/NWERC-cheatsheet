\subsection{Collected Binomials}

\begin{lstlisting}
  //Berechnet alle Binomialkoeffizienten (n ueber k) mod m mit n<N
int binom[N][N];
void calcbinomials(int m) {
	for(int n=0; n<N; n++) {
		binom[n][0] = binom[n][n] = 1;
		for(int k=1; k<n; k++)
			binom[n][k] = (binom[n-1][k]+binom[n-1][k-1])%m;
	}
}
//Berechnet einzelnen Binomialkoeffizienten in Restklasse O(log n)
void calcbinom(int n, int k, int m) {
	return (fak[n] * inverse(fak[k], m) * inverse(fak[n-k], m))%m;
} //fak[n] = (n!)%m


//Berechnet fuer fixes n fuer alle k (n ueber k)  O(n)
void calcbinomrow(int n) {
	binom[n][0] = 1;
	for(int k=1; k<=n; k++) {
		binom[n][k] = binom[n][k-1]*(n-k+1)/k; //*inv(k) % MOD
	}
}
\end{lstlisting}
