\subsection{Graham Scan -- Konvexe Huelle}

        \begin{enumerate}
        \item Finde $p_0$ mit min $y$, Unentschieden: betrachte $x$
        \item Sortiere $p_{1\ldots n}$. $p_i < p_j = ccw(p_0,p_i,p_j)$\\
        (colinear $\rightarrow$ naechster zuerst)
        \item Setze $p_{n+1}=p_0$
        \item Push($p_0$); Push($p_1$); Push($p_2$);
        \item for $i=3$ to $n+1$
        \begin{enumerate}
            \item Solange Winkel der letzten zwei des Stacks und $p_i$ rechtskurve: Pop()
            \item Push($p_i$)
        \end{enumerate}
        \end{enumerate}

\begin{lstlisting}
int minPoint = 0;
for(int i = 1; i < n; ++i)
{
if(points[i].y < points[minPoint].y || (points[i].y == points[minPoint].y && points[i].x < points[minPoint].x))
{
minPoint = i;
}
}
final int mx = points[minPoint].x;
final int my = points[minPoint].y;
Arrays.sort(points, new Comparator<Point>()
{
@Override
public int compare(Point a, Point b) {
int ccw = Line2D.relativeCCW(mx, my, a.x, a.y, b.x, b.y);
if(ccw == 0 || Line2D.relativeCCW(mx, my, b.x, b.y, a.x, a.y) == 0)
{
// gleich...
double d1 = a.distance(mx, my);
double d2 = b.distance(mx, my);
if((d2 < d1 && d2 != 0) || d1 == 0)
{
return 1;
}else
{
return -1;
}
}else if(ccw == 1)
{
// clockwise... -> zuerst b -> a > b
return 1;
}else if(ccw == -1)
{
return -1;
}else
{
System.out.println("shouldnt happen");
System.exit(1);
}
// return 0;
return 0;
}
});

ArrayList<Integer> stack = new ArrayList<Integer>();
stack.add(n-1);
for(int i = 0; i < n; ++i)
{
if(stack.size() < 2)
{
stack.add(i);
continue;
}
int last = stack.get(stack.size() - 1);
int l2 = stack.get(stack.size() - 2);
int ccw = Line2D.relativeCCW(points[l2].x, points[l2].y, points[last].x, points[last].y, points[i].x, points[i].y);
if(ccw != -1)
{
// clockwise oder gleiche Linie
stack.remove(stack.size() - 1);
i--;
}else
{
stack.add(i);
}
}
\end{lstlisting}