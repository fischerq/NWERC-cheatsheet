\begin{lstlisting}[language=C++]
#include <iostream>
#include <iomanip>
#include <fstream>
#include <sstream>
#include <limits>
#include <algorithm>
#include <math.h>
#include <cstdlib>

#include <queue>
#include <vector>
#include <set>
#include <map>
#include <unordered_map>
#include <unordered_set>

using namespace std;
const int iMAX = numeric_limits<int>::max();
const int iMIN = numeric_limits<int>::min();
const double eps = 1e-9;

typedef long long ll;
typedef vector<int> vi;
typedef vector<vector<int>> vii;
typedef pair<int, int> pii;

#define FOR(i,a,b) for(int i = (a); i < (b); i++)
#define all(v) (v).begin(), (v).end()
#define pb push_back
#define mp make_pair

int main() {
	// massively improve cout and cin performance for large streams
	ios::sync_with_stdio(false);
	cin.tie(0);
	
	// Ouput a specific number of digits past the decimal point, in this case 5    
	cout.setf(ios::fixed); cout << setprecision(5);
	cout << 100.0/7.0 << endl;
	cout.unsetf(ios::fixed);
    
	// Output the decimal point and trailing zeros
	cout.setf(ios::showpoint);
	cout << 100.0 << endl;
    
	// Output a '+' before positive values
	cout.setf(ios::showpos);
	cout << 100 << " " << -100 << endl;
    
	// Output numerical values in hexadecimal
	cout << hex << 100 << " " << 1000 << " " << 10000 << dec << endl;
}
\end{lstlisting}